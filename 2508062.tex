\documentclass[uplatex,a4paper,11pt]{jsreport}

\usepackage{array}
\usepackage{geometry}
\geometry{left=25mm,right=10mm,top=15mm,bottom=15mm}
\usepackage{cellspace} % 追加: cellspace パッケージを読み込む
\newcolumntype{S}[1]{>{\cellspace}#1} % S列型を定義

\setlength{\arrayrulewidth}{1pt}

% ↓↓↓ 追加: セルの上下の余白を設定します ↓↓↓
% これにより、セルの内容の高さとこれらの余白の合計がセルの高さになります。
% 目標が10mmの高さで、文字の高さが約4mmとすると、残りの6mmを上下に均等に割り振るため、
% 各3mmずつ追加します。
\cellspacetoplimit=3mm   % セルの上側の余白
\cellspacebottomlimit=3mm % セルの下側の余白
% ↑↑↑ ここまで追加 ↑↑↑

\begin{document}

\section*{魔法陣のような表組}

ご指定のフォーマット(線幅1pt、セル幅10mm、3x3の格子)で、1から9までの整数を配置した表です。

\begin{center}
\setlength{\arrayrulewidth}{1pt}
\begin{tabular}{|S{>{\centering\arraybackslash}p{10mm}}|S{>{\centering\arraybackslash}p{10mm}}|S{>{\centering\arraybackslash}p{10mm}}|}
\hline
8 & 1 & 6 \\
\hline
3 & 5 & 7 \\
\hline
4 & 9 & 2 \\
\hline
\end{tabular}
\end{center}

\vspace{1em}

各セルの内容は、デフォルトでは左寄せになりますが、`>{\centering\arraybackslash}` を列定義に含めることで中央に配置しています。
`p{10mm}` は、「その列のセルに入るテキストの幅が10mm」という意味です。
`\cellspacetoplimit` と `\cellspacebottomlimit` の設定により、セルの上下に余白が追加され、セルの縦寸法が調整されます。

\section*{別の例}
セル内に文字や他の記号を入れる場合も同様です。

\begin{center}
\begin{tabular}{|S{>{\centering\arraybackslash}p{10mm}}|S{>{\centering\arraybackslash}p{10mm}}|S{>{\centering\arraybackslash}p{10mm}}|}
\hline
A & ! & C \\
\hline
\$ & E & 6 \\
\hline
G & 8 & I \\
\hline
\end{tabular}
\end{center}

\end{document}