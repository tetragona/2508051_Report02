\documentclass[
  report,
  a4paper,
  12pt,
  left=25mm,
  right=10mm,
  top=15mm,
  bottom=15mm
]{jlreq}
\usepackage{array}
\usepackage{multirow}
\usepackage{graphicx} % \resizebox用

% 線幅1pt指定用
\makeatletter
\def\hlinewd#1{%
  \noalign{\ifnum0=`}\fi\hrule \@height #1 \futurelet
   \reserved@a\@xhline}
\makeatother

\begin{document}

\chapter{序章}
\chapter{背景}
本研究に至る経過を簡単に記す
\chapter{目的}
実験・シミュレーションの目的を簡潔に記す
\chapter{実験方法/解析方法}
実験・シミュレーションの内容を簡潔に記す
箇条書きもOK
\chapter{結果}
実験・シミュレーションの結果を簡潔に記す
箇条書きもOK
\chapter{考察 }
実験・シミュレーションの考察を簡潔に記す
箇条書きもOK
\chapter{結論}
結論を記す
目的と結論が一対になっているかを確認する
箇条書きもOK
\chapter{今後の進め方}
   今後の進め方を記す
   箇条書きもOK
\chapter{参考報告書・文献}
   関係する報告書・文献を記す



\setlength{\tabcolsep}{0pt} % セル内余白を0に
\renewcommand{\arraystretch}{1.4} % 行の高さ調整用
\newcommand{\cellsize}{10mm}

\begin{center}
\begin{tabular}{|>{\centering\arraybackslash}m{\cellsize}|>{\centering\arraybackslash}m{\cellsize}|>{\centering\arraybackslash}m{\cellsize}|}
\hlinewd{1pt}
 1 & 2 & 3 \\ \hlinewd{1pt}
 4 & 5 & 6 \\ \hlinewd{1pt}
 7 & 8 & 9 \\ \hlinewd{1pt}
\end{tabular}
\end{center}

\end{document}